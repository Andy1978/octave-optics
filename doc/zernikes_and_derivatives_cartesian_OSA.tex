%% LyX 2.3.3 created this file.  For more info, see http://www.lyx.org/.
%% Do not edit unless you really know what you are doing.
\documentclass[english]{article}
\usepackage[T1]{fontenc}
\usepackage[latin9]{inputenc}
\usepackage[a4paper]{geometry}
\geometry{verbose,tmargin=2.5cm,bmargin=2.5cm,lmargin=2.5cm,rmargin=2.5cm}
\usepackage{color}
\usepackage{babel}
\usepackage{array}
\usepackage{amsmath}
\usepackage[unicode=true,pdfusetitle,
 bookmarks=true,bookmarksnumbered=false,bookmarksopen=false,
 breaklinks=false,pdfborder={0 0 0},pdfborderstyle={},backref=false,colorlinks=true]
 {hyperref}

\makeatletter

%%%%%%%%%%%%%%%%%%%%%%%%%%%%%% LyX specific LaTeX commands.
%% Because html converters don't know tabularnewline
\providecommand{\tabularnewline}{\\}

\makeatother

\begin{document}
\title{Calculation of the Zernike polynomials and their partial derivatives
in Cartesian coordinates using the standard indexes OSA/ANSI.}
\author{Jos� Ramom Flores das Seixas}
\date{August 4, 2019}

\maketitle
\tableofcontents{}

\section{Introduction}

The \texttt{zernikes\_and\_derivatives\_cartesian\_OSA} function calculates
the Zernike polynomials and their partial derivatives in Cartesian
coordinates. For this purpose, it uses the algorithm described in
reference \cite{Andersen_2018}. However, it should be noted that
in the cited article the author use unit-normalized Zernike polynomials
arranged according to the azimuthal scheme set forth by Rimmer and
Wyant \cite{Rimmer=000026Wyant_1975}, while to implement the function
has been used the OSA/ANSI standard notation described in references
\cite{Thibos_et_al_2002,ANSI-2017}. This means the index scheme is
different and, moreover, the polynomials are not unit-normalized but
normalized to $\pi$.

\section{Notation}

Zernike's polynomials are usually ordered by a double-index, $Z_{n}^{m}$,
being $n$ the radial order and $m$ the angular frequency, both integers.
Although in programming it is usually utilized a single index, $Z_{j}$.

The scheme used in reference \cite{Andersen_2018}, consider a double-index
with $n\geq0$ and $0\leq m\leq n$. However, the standard OSA/ANSI\footnote{In fact, this double index schema is the most usual, and is employed
by other notations than the standard OSA/ANSI.} uses a double-index with $n\geq0$ and $-n\leq m\leq n$, as shown
in the following table:
\noindent \begin{center}
\begin{tabular}{|c|c|c|c|>{\centering}p{3cm}|}
\multicolumn{2}{|c|}{Rimmer\&Wyant schema} &  & \multicolumn{2}{c|}{~~~~OSA standard~~~~}\tabularnewline
\cline{1-2} \cline{2-2} \cline{4-5} \cline{5-5} 
$\tilde{n}$ & $\tilde{m}$ &  & $n$ & $m$\tabularnewline
\cline{1-2} \cline{2-2} \cline{4-5} \cline{5-5} 
0 & 0 &  & 0 & 0\tabularnewline
\cline{1-2} \cline{2-2} \cline{4-5} \cline{5-5} 
1 & 0, 1 &  & 1 & -1, 1\tabularnewline
\cline{1-2} \cline{2-2} \cline{4-5} \cline{5-5} 
2 & 0, 1, 2 &  & 2 & -2, 0, 2\tabularnewline
\cline{1-2} \cline{2-2} \cline{4-5} \cline{5-5} 
3 & 0, 1, 2, 3 &  & 3 & -3, -1, 1, 3\tabularnewline
\cline{1-2} \cline{2-2} \cline{4-5} \cline{5-5} 
4 & 0, 1, 2, 3, 4 &  & 4 & -4, -2, 0, 2, 4\tabularnewline
\cline{1-2} \cline{2-2} \cline{4-5} \cline{5-5} 
$\vdots$ & $\vdots$ &  & $\vdots$ & $\vdots$\tabularnewline
\cline{1-2} \cline{2-2} \cline{4-5} \cline{5-5} 
\end{tabular}
\par\end{center}

Therefore, by the way of example, the following polynomials --using
the two schemes seen-- are equivalent:
\noindent \begin{center}
\begin{tabular}{ccc}
Rimmer\&Wyant schema &  & ~~~~OSA standard~~~~\tabularnewline
\cline{1-1} \cline{3-3} 
$Z_{1}^{0}\left(x,y\right)$ & ~~~~ & $Z_{1}^{-1}\left(x,y\right)$\tabularnewline
$Z_{3}^{0}\left(x,y\right)$ &  & $Z_{3}^{-3}\left(x,y\right)$\tabularnewline
$Z_{3}^{1}\left(x,y\right)$ &  & $Z_{3}^{-1}\left(x,y\right)$\tabularnewline
$Z_{4}^{4}\left(x,y\right)$ &  & $Z_{4}^{4}\left(x,y\right)$\tabularnewline
$Z_{8}^{6}\left(x,y\right)$ &  & $Z_{8}^{4}\left(x,y\right)$\tabularnewline
\end{tabular}
\par\end{center}

Using $\left(\tilde{n},\tilde{m}\right)$ to denote the R\&W indices,
and $\left(n,m\right)$ those of the OSA scheme, both pairs of indices
are related as follow:
\begin{equation}
n=\tilde{n}\qquad m=2\cdot\tilde{m}-\tilde{n}\label{eq:R=000026W_to_OSA}
\end{equation}


\subsection{Single index}

As said before, it is more convenient to use a simple index to program.
In the OSA standard, the conversion of indices is \href{https://en.wikipedia.org/wiki/Zernike_polynomials\#OSA/ANSI_standard_indices}{as follows}:
\[
j=\frac{n\left(n+2\right)+m}{2}\qquad\begin{array}{l}
n=roundup\left[\dfrac{-3+\sqrt{9+8\,j}}{2}\right]\\
\\
m=2\,j-n\left(n+2\right)
\end{array}
\]


\subsection{U-polynomials}

Employing the notation used in reference \cite{Andersen_2018}, the
unit-normalized Zernike's polynomials are called $U_{nm}$, and are
related to the $\pi$-normalized ones as follows:
\[
Z_{n}^{m}\left(x,y\right)=N_{nm}\cdot U_{nm}\left(x,y\right)
\]
 with
\[
N_{nm}=\sqrt{\frac{2\left(n+1\right)}{1+\delta_{m0}}}\qquad\textrm{being}\quad\delta_{m0}=\left\{ \begin{array}{ll}
1 & m=0\\
0 & m\neq0
\end{array}\right.
\]


\section{Algorithm}

It is well known that for high-order Zernike's polynomials, computing
the values using explicit expressions is not the best strategy, since
it is inefficient and suffers form large cancellation errors. So,
several schemes using recurrence relations have been devised.

Reference \cite{Andersen_2018} presents one recurrence relation with
coefficients that do not depend on radial or azimuthal orders and
which contains no singularities. In addition, it also presents a recurrence
relation to compute the partial derivatives in Cartesian coordinates.

\subsection{Recurrence relations in Cartesian coordinates for Zernike U-polynomials}

The general recurrence relation is
\[
U_{\tilde{n},\tilde{m}}=xU_{\tilde{n}-1,\tilde{m}}+yU_{\tilde{n}-1,\tilde{n}-1-\tilde{m}}+xU_{\tilde{n}-1,\tilde{m}-1}-yU_{\tilde{n}-1,\tilde{n}-\tilde{m}}-U_{\tilde{n}-2,\tilde{m}-1}
\]
But there are several exceptions:
\begin{itemize}
\item for $\tilde{m}=0$$\qquad U_{\tilde{n},0}=xU_{\tilde{n}-1,0}+yU_{\tilde{n}-1,\tilde{n}-1}$
\item for $\tilde{m}=\tilde{n}$$\qquad U_{\tilde{n},\tilde{n}}=xU_{\tilde{n}-1,\tilde{n}-1}-yU_{\tilde{n}-1,0}$
\item for $\tilde{n}$ odd and $\tilde{m}=\dfrac{\tilde{n}-1}{2}$
\[
U_{\tilde{n},\tilde{m}}=yU_{\tilde{n}-1,\tilde{n}-1-\tilde{m}}+xU_{\tilde{n}-1,\tilde{m}-1}-yU_{\tilde{n}-1,\tilde{n}-\tilde{m}}-U_{\tilde{n}-2,\tilde{m}-1}
\]
\item for $\tilde{n}$ odd and $\tilde{m}=\dfrac{\tilde{n}-1}{2}+1$
\[
U_{\tilde{n},\tilde{m}}=xU_{\tilde{n}-1,\tilde{m}}+yU_{\tilde{n}-1,\tilde{n}-1-\tilde{m}}+yU_{\tilde{n}-1,\tilde{m}-1}-U_{\tilde{n}-2,\tilde{m}-1}
\]
\item for $\tilde{n}$ even and $\tilde{m}=\dfrac{\tilde{n}}{2}$
\[
U_{\tilde{n},\tilde{m}}=2xU_{\tilde{n}-1,\tilde{m}}+2yU_{\tilde{n}-1,\tilde{m}-1}-U_{\tilde{n}-2,\tilde{m}-1}
\]
\end{itemize}
The starting polynomials being
\[
U_{0,0}=1,\quad U_{1,0}=y,\quad U_{1,1}=x\quad.
\]


\subsection{Recurrence relations for the OSA/ANSI scheme}

If we now ``translate'' the previous equations into the OSA scheme,
we obtain that the general recurrence relation is
\[
U_{n.m}=xU_{n-1,m+1}+yU_{n-1,-\left(m+1\right)}+xU_{n-1,m-1}-yU_{n-1,1-m}-U_{n-2,m}
\]
the exceptions are given by
\begin{itemize}
\item for $m=-n$$\qquad U_{n,-n}=xU_{n-1,1-n}+yU_{n-1,n-1}$
\item for $m=n$$\qquad U_{n,n}=xU_{n-1,n-1}-yU_{n-1,1-n}$
\item for $m=-1$$\qquad U_{n,-1}=yU_{n-1,0}+xU_{n-1,-2}-yU_{n-1,2}-U_{n-2,-1}$
\item for $m=1$$\qquad U_{n,1}=xU_{n-1,2}+yU_{n-1,-2}+xU_{n-1,0}-U_{n-2,1}$
\item for $m=0$$\qquad U_{n,0}=2xU_{n-1,1}+2yU_{n-1,-1}-U_{n-2,0}$
\end{itemize}
and the starting polynomials are
\[
U_{0,0}=1,\quad U_{1,-1}=y\quad U_{1,1}=x\quad.
\]

Note that using the OSA scheme the conditions of the exceptions have
been simplified, which makes it easier to compute.

\subsubsection{Recurrence relations with a single index}

As mentioned earlier, when programming it is more convenient to use
a simple index. In this case, recurrence relations are given as follows.
Let us start now with the exceptions:
\begin{itemize}
\item for $m=-n\quad\Rightarrow\quad j=\dfrac{\left(n+1\right)n}{n}$
\[
U_{j}=xU_{j-n}+yU_{j-1}
\]
\item for $m=n\quad\Rightarrow\quad j=\dfrac{n\left(n+3\right)}{2}$
\[
U_{j}=xU_{j-\left(n+1\right)}-yU_{j-2n}
\]
\item for $m=-1\quad\Rightarrow\quad j=\dfrac{n\left(n+2\right)-1}{2}$
\[
U_{j}=xU_{j-\left(n+1\right)}+yU_{j-n}-yU_{j-\left(n-1\right)}-U_{j-2n}
\]
\item for $m=1\quad\Rightarrow\quad j=\dfrac{n\left(n+2\right)+1}{2}$
\[
U_{j}=xU_{j-n}+xU_{j-\left(n+1\right)}+yU_{j-\left(n+2\right)}-U_{j-2n}
\]
\item for $m=0\quad\Rightarrow\quad j=\dfrac{n\left(n+2\right)}{2}$
\[
U_{j}=2xU_{j-n}+2yU_{j-\left(n+1\right)}-U_{j-2n}\,.
\]
\end{itemize}
The general case is
\[
U_{j}=xU_{j-n}+yU_{j-\left(n+m+1\right)}+xU_{j-\left(n+1\right)}-yU_{j-\left(m+m\right)}-U_{j-2n}
\]
and the starting polynomials are
\[
U_{0}=1,\quad U_{1}=y,\quad U_{2}=x\quad.
\]


\subsection{Recurrence relations for Cartesian derivatives for Zernike U-polynomials}

The general recursive relations for partial derivatives are:
\[
\frac{\partial U_{\tilde{n},\tilde{m}}}{\partial x}=\tilde{n}U_{\tilde{n}-1,\tilde{m}}+\tilde{n}U_{\tilde{n}-1,\tilde{m}-1}+\frac{\partial U_{\tilde{n}-2,\tilde{m}-1}}{\partial x}
\]
and
\[
\frac{\partial U_{\tilde{n},\tilde{m}}}{\partial y}=\tilde{n}U_{\tilde{n}-1}-\tilde{n}U_{\tilde{n}-1,\tilde{n}-\tilde{m}}+\frac{\partial U_{\tilde{n}-2,\tilde{m}-1}}{\partial y}
\]
But, as before, there are several exceptions:
\begin{itemize}
\item for $\tilde{m}=0$
\[
\frac{\partial U_{\tilde{n},0}}{\partial x}=\tilde{n}U_{\tilde{n}-1,0}\qquad\frac{\partial U_{\tilde{n},0}}{\partial y}=\tilde{n}U_{\tilde{n}-1,\tilde{n}-1}
\]
\item for $\tilde{m}=\tilde{n}$
\[
\frac{\partial U_{\tilde{n},\tilde{n}}}{\partial x}=-\tilde{n}U_{\tilde{n}-1,\tilde{n}-1}\qquad\frac{\partial U_{\tilde{n},\tilde{n}}}{\partial y}=-\tilde{n}U_{\tilde{n}-1,0}
\]
\item for $\tilde{n}$ odd and $\tilde{m}=\dfrac{\tilde{n}-1}{2}$
\[
\frac{\partial U_{\tilde{n},\tilde{m}}}{\partial x}=\tilde{n}U_{\tilde{n}-1,\tilde{m}-1}+\frac{\partial U_{\tilde{n}-2,\tilde{m}-1}}{\partial x}\qquad\frac{\partial U_{\tilde{n},\tilde{m}}}{\partial y}=\tilde{n}U_{\tilde{n}-1,\tilde{n}-\tilde{m}-1}-\tilde{n}U_{\tilde{n}-1,\tilde{n}-\tilde{m}}+\frac{\partial U_{\tilde{n}-2,\tilde{m}-1}}{\partial y}
\]
\item for $\tilde{n}$ odd and $\tilde{m}=\dfrac{\tilde{n}-1}{2}+1$
\[
\frac{\partial U_{\tilde{n},\tilde{m}}}{\partial x}=\tilde{n}U_{\tilde{n}-1,\tilde{m}}+\tilde{n}U_{\tilde{n}-1,\tilde{m}-1}+\frac{\partial U_{\tilde{n}-2,\tilde{m}-1}}{\partial x}\qquad\frac{\partial U_{\tilde{n},\tilde{m}}}{\partial y}=\tilde{n}U_{\tilde{n}-1,\tilde{n}-\tilde{m}-1}+\frac{\partial U_{\tilde{n}-2,\tilde{m}-1}}{\partial y}
\]
\item for $\tilde{n}$ even and $\tilde{m}=\dfrac{\tilde{n}}{2}$
\[
\frac{\partial U_{\tilde{n},\tilde{m}}}{\partial x}=2\tilde{n}U_{\tilde{n}-1,\tilde{m}}+\frac{\partial U_{\tilde{n}-2,\tilde{m}-1}}{\partial x}\qquad\frac{\partial U_{\tilde{n},\tilde{m}}}{\partial y}=2\tilde{n}U_{\tilde{n}-1,\tilde{n}-\tilde{m}-1}+\frac{\partial U_{\tilde{n}-2,\tilde{m}-1}}{\partial y}
\]
\end{itemize}
The starting expressions for the Cartesian derivatives being:
\[
\frac{\partial U_{0,0}}{\partial x}=\frac{\partial U_{0,0}}{\partial y}=0,\quad\frac{\partial U_{1,0}}{\partial x}=\frac{\partial U_{1,1}}{\partial y}=0,\quad\frac{\partial U_{1,1}}{\partial x}=\frac{\partial U_{1,0}}{\partial y}=1
\]


\subsubsection{Recurrence relations for the OSA/ANSI scheme}

If we now ``translate'' the previous equations into the OSA scheme,
we obtain that the general recurrence relations for partial derivativess
are
\[
\frac{\partial U_{n,m}}{\partial x}=nU_{n-1,m+1}+nU_{n-1,m-1}+\frac{\partial U_{n-2,m}}{\partial x}
\]
and
\[
\frac{\partial U_{n,m}}{\partial y}=nU_{n-1,-\left(m+1\right)}-nU_{n-1,1-m}+\frac{\partial U_{n-2,m}}{\partial y}
\]
the exceptions are give by
\begin{itemize}
\item for $m=-n$$\qquad\dfrac{\partial U_{n,-n}}{\partial x}=nU_{n-1,1-n}\qquad\dfrac{\partial U_{n,-n}}{\partial y}=nU_{n-1,n-1}$
\item for $m=n$$\qquad\dfrac{\partial U_{n,-n}}{\partial x}=nU_{n-1,n-1}\qquad\dfrac{\partial U_{n,-n}}{\partial y}=-nU_{n-1,1-n}$
\item for $m=-1$
\[
\dfrac{\partial U_{n,-1}}{\partial x}=nU_{n-1,-2}+\dfrac{\partial U_{n-2,-1}}{\partial x}\qquad\dfrac{\partial U_{n,-1}}{\partial y}=nU_{n-1,0}-nU_{n-1,2}+\dfrac{\partial U_{n-2,-1}}{\partial y}
\]
\item for $m=1$
\[
\dfrac{\partial U_{n,1}}{\partial x}=nU_{n-1,2}+nU_{n-1,0}+\dfrac{\partial U_{n-2,1}}{\partial x}\qquad\dfrac{\partial U_{n,1}}{\partial y}=nU_{n-1,-2}+\dfrac{\partial U_{n-2,1}}{\partial y}
\]
\item for $m=0$
\[
\dfrac{\partial U_{n,0}}{\partial x}=2nU_{n-1,1}+\dfrac{\partial U_{n-2,0}}{\partial x}\qquad\dfrac{\partial U_{n,0}}{\partial y}=2nU_{n-1,-1}+\dfrac{\partial U_{n-2,0}}{\partial y}
\]
\end{itemize}
and the starting derivatives are
\[
\frac{\partial U_{0,0}}{\partial x}=\frac{\partial U_{0,0}}{\partial y}=0,\quad\frac{\partial U_{1,-1}}{\partial x}=\frac{\partial U_{1,1}}{\partial y}=0,\quad\frac{\partial U_{1,1}}{\partial x}=\frac{\partial U_{1,-1}}{\partial y}=1
\]


\subsubsection{Recurrence relations with a single index}

Let us start now with the exceptions:
\begin{itemize}
\item for $m=-n\quad\Rightarrow\quad j=\dfrac{\left(n+1\right)n}{n}$
\[
\frac{\partial U_{j}}{\partial x}=nU_{j-n}\qquad\frac{\partial U_{j}}{\partial y}=nU_{j-1}
\]
\item for $m=n\quad\Rightarrow\quad j=\dfrac{n\left(n+3\right)}{2}$
\[
\frac{\partial U_{j}}{\partial x}=nU_{j-\left(n+1\right)}\qquad\frac{\partial U_{j}}{\partial y}=-nU_{j-2n}
\]
\item for $m=-1\quad\Rightarrow\quad j=\dfrac{n\left(n+2\right)-1}{2}$
\[
\frac{\partial U_{j}}{\partial x}=nU_{j-\left(n+1\right)}+\frac{\partial U_{j-2n}}{\partial x}\qquad\frac{\partial U_{j}}{\partial y}=nU_{j-n}-nU_{j-\left(n-1\right)}+\frac{\partial U_{j-2n}}{\partial y}
\]
\item for $m=1\quad\Rightarrow\quad j=\dfrac{n\left(n+2\right)+1}{2}$
\[
\frac{\partial U_{j}}{\partial x}=nU_{j-n}+nU_{j-\left(n+1\right)}+\frac{\partial U_{j-2n}}{\partial x}\qquad\frac{\partial U_{j}}{\partial y}=nU_{j-\left(n+2\right)}+\frac{\partial U_{j-2n}}{\partial y}
\]
\item for $m=0\quad\Rightarrow\quad j=\dfrac{n\left(n+2\right)}{2}$
\[
\frac{\partial U_{j}}{\partial x}=2nU_{j-n}+\frac{\partial U_{j-2n}}{\partial x}\qquad\frac{\partial U_{j}}{\partial y}=2nU_{j-\left(n+1\right)}+\frac{\partial U_{j-2n}}{\partial y}
\]
\end{itemize}
The general case is
\[
\frac{\partial U_{j}}{\partial x}=nU_{j-n}+nU_{j-\left(n+1\right)}+\frac{\partial U_{j-2n}}{\partial x}\qquad\frac{\partial U_{j}}{\partial y}=nU_{j-\left(n+m+1\right)}-nU_{j-\left(n+m\right)}+\frac{\partial U_{j-2n}}{\partial y}
\]
and the starting derivatives are

\[
\frac{\partial U_{0}}{\partial x}=\frac{\partial U_{0}}{\partial y}=0,\quad\frac{\partial U_{1}}{\partial x}=\frac{\partial U_{2}}{\partial y}=0,\quad\frac{\partial U_{2}}{\partial x}=\frac{\partial U_{1}}{\partial y}=1
\]

\begin{thebibliography}{1}
\bibitem{Andersen_2018}Andersen T.B., \href{https://doi.org/10.1364/OE.26.018878}{Efficient and robust recurrence relations for the Zernike circle polynomials and their derivatives in Cartesian coordinates}.
\emph{Optic Express 26}(15), 18878-18896 (2018).

\bibitem{Rimmer=000026Wyant_1975}Rimmer M.P. \& Wyant J.C., \href{https://wp.optics.arizona.edu/jcwyant/wp-content/uploads/sites/13/2016/08/LSI-Analysis.pdf}{Evaluation of large aberrations using a lateral-shear interferometer having variable shear}.
\emph{Appl. Opt. 14}(1), 142--150 (1975).

\bibitem{Thibos_et_al_2002}Thibos, L.N, Applegate, R.A., Schwiegerling,
J.T. \& Webb, R., \href{https://pdfs.semanticscholar.org/6e71/c63afa9960dc6e452f3ae2c8b20e2185ba88.pdf}{Standards for reporting the optical aberrations of eyes}.
\emph{Journal of refractive surgery}, \emph{18}(5), S652-S660 (2002).

\bibitem{ANSI-2017}ANSI Z80.28--2017. American National Standards
of Ophthalmics:methods for reporting optical aberrations of eyes.
\end{thebibliography}

\end{document}
